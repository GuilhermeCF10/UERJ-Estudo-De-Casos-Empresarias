\begin{titlepage}
    \thispagestyle{empty} % Remove números de página
    \setstretch{1.5} % Espaçamento entre linhas, certifique-se de que o pacote setspace está incluído em document.tex

    \begin{center}
        \textbf{\Large RESUMO}
    \end{center}


    \vspace{1cm} % Espaço vertical

    \noindent CAGIDE FIALHO, G. Relatório do projeto de Modelagem
    e controle de sistemas. 2024. 54 f. Trabalho de Conclusão de Disciplina Modelagem e Controle de Sistemas (Graduação em
    Engenharia da computação) – Graduação em Engenharia da Computação, Universidade
    do Estado do Rio de Janeiro, Nova Friburgo, 2024.

    \vspace{0.4cm} % Espaço vertical

    Este trabalho explora a modelagem e o controle de um sistema dinâmico do tipo massa-mola-amortecedor, utilizando a plataforma Scilab para desenvolvimento e simulação. O foco do estudo está na implementação de modelos matemáticos para descrever a dinâmica do sistema e na análise de sua resposta sob diversas condições iniciais, sem a aplicação de forças externas. Utilizando a ferramenta Xcos, um componente gráfico do Scilab, realizamos simulações que permitiram uma análise visual e quantitativa das respostas transientes do sistema. O estudo destaca a influência dos parâmetros físicos, como a massa, o coeficiente de amortecimento, e a constante da mola, nas características de resposta do sistema. Além disso, técnicas de controle foram empregadas para ajustar a resposta do sistema, demonstrando como o amortecimento pode contribuir para a estabilização após perturbações e enfatizando a relevância de uma parametrização cuidadosa para alcançar um comportamento eficaz do sistema. Este projeto contribui para a compreensão das teorias de controle aplicáveis em sistemas mecânicos e outros contextos de sistemas dinâmicos na engenharia.
    \vspace{0.4cm} % Espaço vertical

    \textbf{Palavras-chave}: Modelagem e Controle, Sistema Massa-Mola-Amortecedor, Simulação, Scilab, Xcos.
\end{titlepage}

\section{Esta empresa possui política de preservação do meio ambiente e sustentabilidade? Quais?}

A \textbf{Netflix} adota uma política robusta de preservação ambiental e sustentabilidade como parte de suas práticas de governança ambiental, social e corporativa (ESG). Entre as principais iniciativas, destacam-se:

\begin{itemize}
    \item \textbf{Redução de emissões de carbono}: A Netflix se comprometeu a reduzir suas emissões de carbono em aproximadamente 50\% até 2030. Desde 2022, a empresa compensa as emissões restantes através de investimentos em soluções climáticas naturais verificadas, apoiando as metas globais de emissões líquidas zero \cite{netflix_sustainability}.

    \item \textbf{Sustentabilidade nas produções}: A empresa tem implementado tecnologias sustentáveis em suas produções, como o uso de unidades de energia a hidrogênio e equipamentos elétricos em vez de geradores a diesel. Essas práticas foram testadas em produções como \textit{Bridgerton} e resultaram na redução significativa de emissões de CO2.

    \item \textbf{Investimentos em infraestrutura sustentável}: A Netflix também realiza melhorias em seus estúdios e escritórios, como a instalação de sistemas geotérmicos para aquecimento e resfriamento em seu campus em Albuquerque, com o objetivo de eliminar ou reduzir as emissões das operações prediais.
    
    \item \textbf{Coleção de Histórias Sustentáveis}: A Netflix promove a conscientização ambiental através de seu conteúdo, com uma coleção dedicada a histórias sobre sustentabilidade, que inclui mais de 200 títulos assistidos por milhões de lares ao redor do mundo.

\end{itemize}
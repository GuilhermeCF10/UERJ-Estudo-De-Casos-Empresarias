\section{Esta empresa possui política de preservação do meio ambiente e sustentabilidade? Quais?}

Sim, a Amazon possui uma série de políticas e programas de sustentabilidade para reduzir seu impacto ambiental e promover a preservação do meio ambiente. Entre os principais compromissos estão:

\begin{itemize}
    \item \textbf{The Climate Pledge}: A Amazon se comprometeu a alcançar emissões líquidas de carbono zero até 2040, 10 anos antes do previsto no Acordo de Paris. Como parte desse compromisso, a empresa visa operar com 100\% de energia renovável até 2025. Em 2023, a Amazon já conseguiu alimentar 100\% de suas operações globais com fontes renováveis.

    \item \textbf{Redução de resíduos e embalagens}: A Amazon tem trabalhado para reduzir o uso de embalagens plásticas descartáveis e aumentar a reciclagem. Desde 2015, a empresa diminuiu o peso das embalagens em 43\%, evitando mais de 3 milhões de toneladas de resíduos. Na Europa, por exemplo, a empresa substituiu o plástico descartável por embalagens de papel 100\% recicláveis.

    \item \textbf{Frota de veículos elétricos}: A Amazon está eletrificando sua frota de entregas com o objetivo de reduzir as emissões de carbono. Em 2023, a empresa entregou mais de 680 milhões de pacotes usando veículos elétricos e planeja expandir esse número nos próximos anos.

    \item \textbf{Preservação dos recursos hídricos}: A Amazon, por meio da AWS, estabeleceu uma meta de se tornar "positiva em água" até 2030, ou seja, devolver mais água às comunidades do que consome em suas operações diretas. Esse compromisso faz parte de uma série de iniciativas voltadas para o uso responsável dos recursos hídricos.
\end{itemize}

Essas iniciativas demonstram o compromisso contínuo da Amazon com a sustentabilidade, integrando práticas ambientais responsáveis em suas operações globais \cite{amazonSustainability2023, amazonGlobalImpact2023}.


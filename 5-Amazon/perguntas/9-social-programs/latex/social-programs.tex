\section{Esta empresa possui programas sociais para as comunidades nas quais atua? Quais?}

Sim, a Amazon possui uma série de programas sociais focados em apoiar as comunidades onde a empresa opera. Esses programas estão centrados em áreas como educação, moradia, segurança alimentar, e resposta a desastres. Alguns dos principais programas são:

\begin{itemize}
    \item \textbf{Amazon Future Engineer}: Um programa global que oferece educação em ciência da computação, com foco em estudantes de comunidades sub-representadas. A iniciativa já impactou milhões de estudantes em mais de 6.000 escolas, oferecendo cursos de STEM e bolsas de estudo.

    \item \textbf{Housing Equity Fund}: A Amazon investe em moradias acessíveis por meio deste fundo, que já destinou mais de \$2 bilhões para criar e preservar mais de 21.000 unidades de moradia acessível nos Estados Unidos, principalmente em comunidades de baixa renda e minorias.

    \item \textbf{Resposta a desastres}: A Amazon mantém nove centros de alívio de desastres ao redor do mundo, equipados com itens essenciais para apoiar comunidades afetadas por desastres naturais. Em 2023, a empresa respondeu a mais de 115 desastres, enviando milhões de itens de ajuda.

    \item \textbf{Segurança alimentar}: A empresa colabora com bancos de alimentos e organizações locais para enfrentar a insegurança alimentar, tendo doado mais de 88 milhões de libras de alimentos e entregue mais de 33 milhões de refeições diretamente a famílias necessitadas.

    \item \textbf{AWS Skills Centers e AWS re/Start}: Esses programas oferecem treinamento gratuito em computação em nuvem para comunidades desfavorecidas, ajudando pessoas a desenvolver habilidades tecnológicas e conectando-as com oportunidades de emprego.
\end{itemize}

Esses programas refletem o compromisso da Amazon em apoiar o desenvolvimento social e econômico das comunidades onde atua, com foco em educação, inclusão e ajuda humanitária.\cite{amazonSustainability}


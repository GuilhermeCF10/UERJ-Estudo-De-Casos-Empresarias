\section{Esta empresa possui política de preservação do meio ambiente e sustentabilidade? Quais?}

Sim, o iFood possui uma política robusta de preservação ambiental e sustentabilidade, com metas claras para reduzir o impacto ambiental até 2025. Entre as principais iniciativas estão:

\begin{itemize}
    \item \textbf{Redução de Plástico}: O iFood estabeleceu o compromisso de zerar a poluição plástica em suas operações de entrega. Isso inclui a redução do uso de plásticos de uso único (como talheres e canudos), substituição por embalagens sustentáveis e incentivo à reciclagem.
    
    \item \textbf{Neutralidade de Carbono}: Desde 2021, o iFood é neutro em emissão de carbono em suas entregas. A empresa também busca realizar ao menos 50\% de suas entregas com modais não poluentes, como bicicletas e motos elétricas \cite{ifoodmeioambiente2023}.
    
    \item \textbf{iFood Pedal}: Programa de aluguel de bicicletas para entregadores, incentivando o uso de modais não poluentes, o que já representa 22\% dos pedidos entregues por meio de bicicletas.
    
    \item \textbf{Reciclagem e Embalagens Sustentáveis}: Parcerias com empresas como Suzano e Klabin para o desenvolvimento de embalagens recicláveis. O programa \textbf{Amigos da Natureza} incentiva os consumidores a dispensarem o uso de plásticos descartáveis, com a adesão de mais de 90\% dos restaurantes no app \cite{ifoodrelatorio2023}.
    
    \item \textbf{Restauração Florestal}: Investimentos em projetos de reflorestamento, como a parceria com a ONG SOS Mata Atlântica, que já plantou 50 mil mudas de árvores nativas \cite{ifoodnews2023}.
\end{itemize}
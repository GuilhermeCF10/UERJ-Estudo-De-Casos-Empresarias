% ===============================================================
% Document Class ================================================
\documentclass{article}

% ===============================================================
% Graphic Packages ==============================================
\usepackage{graphicx}       % Permite a inclusão de imagens no documento.
\usepackage{tocloft} % Pacote para customização de listas de conteúdos, figuras e tabelas
\usepackage{tikz}           % Ferramenta poderosa para criar gráficos programaticamente dentro do LaTeX.
\usetikzlibrary{calc}       % Extensão da biblioteca TikZ que permite cálculos mais complexos de coordenadas.
\usepackage[portuguese]{babel}
\usepackage{tocloft} % Pacote para customização de listas
\renewcommand{\listfigurename}{Lista de Figuras} % Altera o título para português
\AtBeginDocument{\renewcommand{\contentsname}{Sumário}} % Altera nome de conteudo para sumário
% ===============================================================
% Mathematical Tools ============================================
\usepackage{amsmath}        % Melhora a aparência e a flexibilidade de comandos matemáticos.
\usepackage{siunitx}        % Facilita o uso de unidades do Sistema Internacional e ajuda a formatar números complexos.

% ===============================================================
% Font and Text Appearance ======================================
\usepackage{mathptmx}       % Altera a fonte padrão do documento para Times New Roman.

% ===============================================================
% Table of Contents Customization ===============================
\usepackage{tocloft}  % Oferece controle total sobre a aparência das listas de conteúdos, figuras, tabelas, etc.

% ===============================================================
% Figure Positioning ============================================
\usepackage{float}     % Melhora a interface para definir o posicionamento de objetos flutuantes como figuras e tabelas.

% ===============================================================
% Paragraph Spacing and Indentation =============================
\usepackage{setspace}  % Permite o ajuste fino do espaçamento entre linhas.
\usepackage{indentfirst} % Adiciona indentação ao primeiro parágrafo de cada seção.

% ===============================================================
% Page Layout ===================================================
\usepackage[a4paper, top=3cm, bottom=2cm, left=3cm, right=2cm]{geometry}  % Define as margens de todo o documento.
\setlength{\parindent}{4em}  % Define o tamanho da indentação para todos os parágrafos.
\setlength{\emergencystretch}{3em}

% ===============================================================
% Section Heading Customization =================================
\makeatletter
\renewcommand\paragraph{\@startsection{paragraph}{4}{\z@}%
    {2ex plus 1ex minus .2ex}%
    {1em}%
    {\normalfont\normalsize\bfseries}}
\makeatother

% ===============================================================
% References =================================
\usepackage{cite} 


% ===============================================================
% Urls ==========================================================
\usepackage{hyperref} % Para criar links clicáveis
\hypersetup{
    colorlinks=true,
    linkcolor=blue,
    filecolor=magenta,
    urlcolor=blue,
    citecolor=blue,
    pdfborder={0 0 0},  % Remove o quadrado ao redor dos links
    breaklinks=true  % Permite que os links sejam quebrados entre as linhas
}
\usepackage{xurl} % Melhora a quebra de links em qualquer lugar

% ===============================================================
% Bloco de código ===============================================
\usepackage{listings}
\usepackage{xcolor}
\usepackage{courier}
\lstset{
  backgroundcolor=\color[RGB]{249,246,239},
  basicstyle=\ttfamily\footnotesize,
  breaklines=true,
  frame=single,
  numbers=left,
  numberstyle=\tiny\color{gray},
  keywordstyle=\color[RGB]{40,40,255},
  commentstyle=\color[RGB]{0,125,0},
  stringstyle=\color[RGB]{255,0,0},
  showstringspaces=false,
  rulecolor=\color{black},
  captionpos=b,
  abovecaptionskip=5pt,
  belowcaptionskip=5pt,
  xleftmargin=0.15\textwidth,
  xrightmargin=0.15\textwidth,
  morecomment=[l]{//}
}
\renewcommand{\lstlistlistingname}{Lista de Códigos} % Comando para renomear o título da Lista de Códigos

% ===============================================================
% Begin Document ================================================
\begin{document}

% Capa basica do Ratamero
\begin{flushright}
  \textbf{Trabalho da Disciplina Estudo de Casos Empresariais} \\
  Orientador: Prof. Leandro de Amorim Ratamero
\end{flushright}

% Título da atividade
{\large\bfseries Trabalho 7 - Ifood \par}

% Nome do estudante
{\large Guilherme Cagide Fialho \par}

% ===============================================================
% Capa ==========================================================
% \input{header/capa/latex/capa.tex}

% ===============================================================
% Contra Capa ===================================================
% \input{header/contra-capa/latex/contra-capa.tex}

% ===============================================================
% Resumo ========================================================
% \section{Conclusão}

Desse modo, como abordado no Brasil, existem várias formas de estruturação empresarial, cada uma com suas especificidades e vantagens, adequadas para diferentes perfis de empreendedores e negócios, como um resumo podemos deixar que:
\begin{itemize}
    \item \textbf{MEI (Microempreendedor Individual)}:
          \begin{itemize}
              \item Permite a formalização de atividades com faturamento anual de até R\$ 81.000,00.
              \item Oferece benefícios como emissão de notas fiscais e contribuição simplificada ao INSS.
              \item Ideal para pequenos negócios e autônomos que buscam formalização sem grandes custos e burocracia.
          \end{itemize}

    \item \textbf{EI (Empresário Individual)}:
          \begin{itemize}
              \item Não possui limite de faturamento específico.
              \item Oferece maior flexibilidade na contratação de empregados.
              \item Responsabilidade ilimitada do empresário, que responde com seu patrimônio pessoal pelas dívidas da empresa.
          \end{itemize}

    \item \textbf{SLU (Sociedade Limitada Unipessoal)}:
          \begin{itemize}
              \item Oferece segurança da limitação de responsabilidade sem a necessidade de um sócio.
              \item O patrimônio pessoal é protegido.
              \item A empresa pode operar com a flexibilidade de uma sociedade limitada, mas com um único sócio.
          \end{itemize}

    \item \textbf{LTDA (Sociedade Limitada)}:
          \begin{itemize}
              \item Ideal para empresas com mais de um sócio.
              \item O capital social é dividido em cotas, e a responsabilidade dos sócios é limitada ao valor dessas cotas.
              \item Oferece proteção patrimonial e uma gestão coletiva com segurança jurídica.
          \end{itemize}

    \item \textbf{SS (Sociedade Simples)}:
          \begin{itemize}
              \item Voltada para profissionais de atividades intelectuais, científicas, literárias ou artísticas.
              \item Permite a união de esforços sem o rigor empresarial.
              \item A responsabilidade dos sócios pode ser ilimitada, dependendo da configuração.
              \item Registro realizado no Registro Civil de Pessoas Jurídicas, não na Junta Comercial.
          \end{itemize}

    \item \textbf{SA (Sociedade Anônima)}:
          \begin{itemize}
              \item Ideal para grandes negócios que buscam captar recursos por meio de ações.
              \item O capital é dividido em ações, e a responsabilidade dos acionistas é limitada ao valor das ações subscritas.
              \item Pode ser de capital aberto, com ações negociadas em bolsa, ou de capital fechado, com ações restritas a um grupo seleto de investidores.
          \end{itemize}

\end{itemize}
\vspace{0.4cm}
Este progresso desde o MEI até a Sociedade Anônima mostra a diversidade de opções disponíveis para empreendedores no Brasil, permitindo que cada negócio escolha a estrutura que melhor se adapta às suas necessidades e objetivos de crescimento.

\vspace{0.4cm}
Além desses, também existe o \textbf{Inova Simples (IS)}, um regime especial criado pela Lei Complementar nº 167/2019, destinado a startups e empresas de inovação. Este regime facilita a criação, formalização e desenvolvimento de negócios inovadores, permitindo que projetos em fase de testes possam ser formalizados de maneira simplificada, com processos mais ágeis de registro e tributação. 

Um ponto importante é que o Inova Simples permite a captação de recursos ou realização de comercializações experimentais de até R\$ 81.000,00, proporcionando uma estrutura simplificada para o desenvolvimento inicial de startups. No entanto, a empresa deve se adequar a um regime tributário caso suas atividades ultrapassem esse limite, ou quando estiver pronta para sair da fase experimental. O Inova Simples é uma alternativa ideal para empreendedores que desejam explorar ideias inovadoras com um caráter disruptivo ou incremental, contribuindo para o fomento da inovação no Brasil.


% ===============================================================
% Abstract ======================================================
% \input{main/abstract/latex/abstract.tex}

% ===============================================================
% Lista de figuras ==============================================
% % \begin{titlepage}
\newpage
\setstretch{1.5} % Espaçamento entre linhas, certifique-se de que o pacote setspace está incluído em document.tex

\section{Referências}

\vspace{0.5cm} % Espaço 

\noindent WIKIPEDIA. \textbf{Itautec}. Disponível em: \url{https://pt.wikipedia.org/wiki/Itautec}. Acesso em: 24 ago. 2024.


\noindent TECMUNDO. \textbf{Perdidas no tempo: 10 grandes empresas que estiveram no topo}. Disponível em: \url{https://www.tecmundo.com.br/empresas-e-instituicoes/56240-perdidas-tempo-10-grandes-empresas-estiveram-no-topo.htm}. Acesso em: 24 ago. 2024.


\noindent MORCONE. \textbf{Grandes empresas brasileiras que partiram}. Disponível em: \url{https://morcone.com.br/2024/05/08/grandes-empresas-brasileiras-que-partiram/}. Acesso em: 24 ago. 2024.


\noindent UOL ECONOMIA. \textbf{O que aconteceu com as Lojas Brasileiras, rival das Americanas?}. Disponível em: \url{https://economia.uol.com.br/noticias/redacao/2023/01/29/o-que-aconteceu-com-as-lojas-brasileiras-rival-das-americanas.htm}. Acesso em: 24 ago. 2024.


\noindent UOL ECONOMIA. \textbf{Mappin: do auge à falência}. Disponível em: \url{https://economia.uol.com.br/noticias/redacao/2023/05/30/mappin-do-auge-a-falencia.htm}. Acesso em: 24 ago. 2024.


\noindent ESTADÃO. \textbf{Mappin volta ao jogo do varejo após falência em 1999 e reveses na pandemia, mas marca ainda tem apelo}. Disponível em: \url{https://www.estadao.com.br/economia/negocios/mappin-volta-ao-jogo-do-varejo-apos-falencia-em-1999-e-reves-na-pandemia-mas-marca-ainda-tem-apelo/}. Acesso em: 24 ago. 2024.

% ===============================================================
% Lista de códigos ==============================================
% % \begin{titlepage}
\newpage
\setstretch{1.5} % Espaçamento entre linhas, certifique-se de que o pacote setspace está incluído em document.tex

\section{Referências}

\vspace{0.5cm} % Espaço 

\noindent WIKIPEDIA. \textbf{Itautec}. Disponível em: \url{https://pt.wikipedia.org/wiki/Itautec}. Acesso em: 24 ago. 2024.


\noindent TECMUNDO. \textbf{Perdidas no tempo: 10 grandes empresas que estiveram no topo}. Disponível em: \url{https://www.tecmundo.com.br/empresas-e-instituicoes/56240-perdidas-tempo-10-grandes-empresas-estiveram-no-topo.htm}. Acesso em: 24 ago. 2024.


\noindent MORCONE. \textbf{Grandes empresas brasileiras que partiram}. Disponível em: \url{https://morcone.com.br/2024/05/08/grandes-empresas-brasileiras-que-partiram/}. Acesso em: 24 ago. 2024.


\noindent UOL ECONOMIA. \textbf{O que aconteceu com as Lojas Brasileiras, rival das Americanas?}. Disponível em: \url{https://economia.uol.com.br/noticias/redacao/2023/01/29/o-que-aconteceu-com-as-lojas-brasileiras-rival-das-americanas.htm}. Acesso em: 24 ago. 2024.


\noindent UOL ECONOMIA. \textbf{Mappin: do auge à falência}. Disponível em: \url{https://economia.uol.com.br/noticias/redacao/2023/05/30/mappin-do-auge-a-falencia.htm}. Acesso em: 24 ago. 2024.


\noindent ESTADÃO. \textbf{Mappin volta ao jogo do varejo após falência em 1999 e reveses na pandemia, mas marca ainda tem apelo}. Disponível em: \url{https://www.estadao.com.br/economia/negocios/mappin-volta-ao-jogo-do-varejo-apos-falencia-em-1999-e-reves-na-pandemia-mas-marca-ainda-tem-apelo/}. Acesso em: 24 ago. 2024.

% ===============================================================
% Lista de siglas e abreviaturas ================================
% % \begin{titlepage}
\newpage
\setstretch{1.5} % Espaçamento entre linhas, certifique-se de que o pacote setspace está incluído em document.tex

\section{Referências}

\vspace{0.5cm} % Espaço 

\noindent WIKIPEDIA. \textbf{Itautec}. Disponível em: \url{https://pt.wikipedia.org/wiki/Itautec}. Acesso em: 24 ago. 2024.


\noindent TECMUNDO. \textbf{Perdidas no tempo: 10 grandes empresas que estiveram no topo}. Disponível em: \url{https://www.tecmundo.com.br/empresas-e-instituicoes/56240-perdidas-tempo-10-grandes-empresas-estiveram-no-topo.htm}. Acesso em: 24 ago. 2024.


\noindent MORCONE. \textbf{Grandes empresas brasileiras que partiram}. Disponível em: \url{https://morcone.com.br/2024/05/08/grandes-empresas-brasileiras-que-partiram/}. Acesso em: 24 ago. 2024.


\noindent UOL ECONOMIA. \textbf{O que aconteceu com as Lojas Brasileiras, rival das Americanas?}. Disponível em: \url{https://economia.uol.com.br/noticias/redacao/2023/01/29/o-que-aconteceu-com-as-lojas-brasileiras-rival-das-americanas.htm}. Acesso em: 24 ago. 2024.


\noindent UOL ECONOMIA. \textbf{Mappin: do auge à falência}. Disponível em: \url{https://economia.uol.com.br/noticias/redacao/2023/05/30/mappin-do-auge-a-falencia.htm}. Acesso em: 24 ago. 2024.


\noindent ESTADÃO. \textbf{Mappin volta ao jogo do varejo após falência em 1999 e reveses na pandemia, mas marca ainda tem apelo}. Disponível em: \url{https://www.estadao.com.br/economia/negocios/mappin-volta-ao-jogo-do-varejo-apos-falencia-em-1999-e-reves-na-pandemia-mas-marca-ainda-tem-apelo/}. Acesso em: 24 ago. 2024.

% ===============================================================
% Materiais utilizados ==========================================
% % \begin{titlepage}
\newpage
\setstretch{1.5} % Espaçamento entre linhas, certifique-se de que o pacote setspace está incluído em document.tex

\section{Referências}

\vspace{0.5cm} % Espaço 

\noindent WIKIPEDIA. \textbf{Itautec}. Disponível em: \url{https://pt.wikipedia.org/wiki/Itautec}. Acesso em: 24 ago. 2024.


\noindent TECMUNDO. \textbf{Perdidas no tempo: 10 grandes empresas que estiveram no topo}. Disponível em: \url{https://www.tecmundo.com.br/empresas-e-instituicoes/56240-perdidas-tempo-10-grandes-empresas-estiveram-no-topo.htm}. Acesso em: 24 ago. 2024.


\noindent MORCONE. \textbf{Grandes empresas brasileiras que partiram}. Disponível em: \url{https://morcone.com.br/2024/05/08/grandes-empresas-brasileiras-que-partiram/}. Acesso em: 24 ago. 2024.


\noindent UOL ECONOMIA. \textbf{O que aconteceu com as Lojas Brasileiras, rival das Americanas?}. Disponível em: \url{https://economia.uol.com.br/noticias/redacao/2023/01/29/o-que-aconteceu-com-as-lojas-brasileiras-rival-das-americanas.htm}. Acesso em: 24 ago. 2024.


\noindent UOL ECONOMIA. \textbf{Mappin: do auge à falência}. Disponível em: \url{https://economia.uol.com.br/noticias/redacao/2023/05/30/mappin-do-auge-a-falencia.htm}. Acesso em: 24 ago. 2024.


\noindent ESTADÃO. \textbf{Mappin volta ao jogo do varejo após falência em 1999 e reveses na pandemia, mas marca ainda tem apelo}. Disponível em: \url{https://www.estadao.com.br/economia/negocios/mappin-volta-ao-jogo-do-varejo-apos-falencia-em-1999-e-reves-na-pandemia-mas-marca-ainda-tem-apelo/}. Acesso em: 24 ago. 2024.

% ===============================================================
% Sumário =======================================================
% % \begin{titlepage}
\newpage
\setstretch{1.5} % Espaçamento entre linhas, certifique-se de que o pacote setspace está incluído em document.tex

\section{Referências}

\vspace{0.5cm} % Espaço 

\noindent WIKIPEDIA. \textbf{Itautec}. Disponível em: \url{https://pt.wikipedia.org/wiki/Itautec}. Acesso em: 24 ago. 2024.


\noindent TECMUNDO. \textbf{Perdidas no tempo: 10 grandes empresas que estiveram no topo}. Disponível em: \url{https://www.tecmundo.com.br/empresas-e-instituicoes/56240-perdidas-tempo-10-grandes-empresas-estiveram-no-topo.htm}. Acesso em: 24 ago. 2024.


\noindent MORCONE. \textbf{Grandes empresas brasileiras que partiram}. Disponível em: \url{https://morcone.com.br/2024/05/08/grandes-empresas-brasileiras-que-partiram/}. Acesso em: 24 ago. 2024.


\noindent UOL ECONOMIA. \textbf{O que aconteceu com as Lojas Brasileiras, rival das Americanas?}. Disponível em: \url{https://economia.uol.com.br/noticias/redacao/2023/01/29/o-que-aconteceu-com-as-lojas-brasileiras-rival-das-americanas.htm}. Acesso em: 24 ago. 2024.


\noindent UOL ECONOMIA. \textbf{Mappin: do auge à falência}. Disponível em: \url{https://economia.uol.com.br/noticias/redacao/2023/05/30/mappin-do-auge-a-falencia.htm}. Acesso em: 24 ago. 2024.


\noindent ESTADÃO. \textbf{Mappin volta ao jogo do varejo após falência em 1999 e reveses na pandemia, mas marca ainda tem apelo}. Disponível em: \url{https://www.estadao.com.br/economia/negocios/mappin-volta-ao-jogo-do-varejo-apos-falencia-em-1999-e-reves-na-pandemia-mas-marca-ainda-tem-apelo/}. Acesso em: 24 ago. 2024.

% ===============================================================
% Introdução ====================================================
% % \begin{titlepage}
\newpage
\setstretch{1.5} % Espaçamento entre linhas, certifique-se de que o pacote setspace está incluído em document.tex

\section{Referências}

\vspace{0.5cm} % Espaço 

\noindent WIKIPEDIA. \textbf{Itautec}. Disponível em: \url{https://pt.wikipedia.org/wiki/Itautec}. Acesso em: 24 ago. 2024.


\noindent TECMUNDO. \textbf{Perdidas no tempo: 10 grandes empresas que estiveram no topo}. Disponível em: \url{https://www.tecmundo.com.br/empresas-e-instituicoes/56240-perdidas-tempo-10-grandes-empresas-estiveram-no-topo.htm}. Acesso em: 24 ago. 2024.


\noindent MORCONE. \textbf{Grandes empresas brasileiras que partiram}. Disponível em: \url{https://morcone.com.br/2024/05/08/grandes-empresas-brasileiras-que-partiram/}. Acesso em: 24 ago. 2024.


\noindent UOL ECONOMIA. \textbf{O que aconteceu com as Lojas Brasileiras, rival das Americanas?}. Disponível em: \url{https://economia.uol.com.br/noticias/redacao/2023/01/29/o-que-aconteceu-com-as-lojas-brasileiras-rival-das-americanas.htm}. Acesso em: 24 ago. 2024.


\noindent UOL ECONOMIA. \textbf{Mappin: do auge à falência}. Disponível em: \url{https://economia.uol.com.br/noticias/redacao/2023/05/30/mappin-do-auge-a-falencia.htm}. Acesso em: 24 ago. 2024.


\noindent ESTADÃO. \textbf{Mappin volta ao jogo do varejo após falência em 1999 e reveses na pandemia, mas marca ainda tem apelo}. Disponível em: \url{https://www.estadao.com.br/economia/negocios/mappin-volta-ao-jogo-do-varejo-apos-falencia-em-1999-e-reves-na-pandemia-mas-marca-ainda-tem-apelo/}. Acesso em: 24 ago. 2024.

% ===============================================================
% Atividades ====================================================
\section{Qual é o nome da empresa em estudo? Quando foi fundada? Qual o(s) nome(s) do(s) fundador(es)?}

A empresa em estudo é o Nubank, uma fintech brasileira fundada em 2013. Os fundadores do Nubank são David Vélez, um colombiano com experiência no setor financeiro, Cristina Junqueira, uma brasileira com histórico no setor bancário, e Edward Wible, um engenheiro de software norte-americano. A empresa se destacou por oferecer serviços bancários digitais, sem taxas, com foco em uma experiência simplificada para o usuário \cite{nubank2023}. % Completa e Validada
\section{Qual é a formação de cada fundador?}
Getúlio Vargas estudou Direito na Faculdade de Direito de Porto Alegre, atualmente parte da Universidade Federal do Rio Grande do Sul (UFRGS), onde se graduou em 1907. Após a graduação, ele trabalhou como promotor público antes de se dedicar integralmente à política. Vargas foi presidente do Brasil por dois períodos: de 1930 a 1945 e de 1951 a 1954. Durante seu segundo mandato, influenciou diretamente na criação da Petrobras, estabelecida em 1953, como parte de suas políticas desenvolvimentistas e nacionalistas. % Completa e Validadada
\section{Quais são os produtos e/ou serviços mais conhecidos desta empresa no mercado?}

O Nubank oferece uma ampla gama de produtos e serviços financeiros, todos com foco em acessibilidade e simplicidade digital. Entre os mais conhecidos, destacam-se:

\begin{itemize}
    \item \textbf{Cartão de Crédito}: O cartão de crédito sem anuidade foi o primeiro produto da empresa, conhecido por não cobrar taxas e ser totalmente gerenciado por um aplicativo.
    \item \textbf{Conta Digital (NuConta)}: Uma conta de pagamento que permite transferências ilimitadas e gratuitas, pagamento de contas, além de rendimento automático sobre o saldo.
    \item \textbf{Empréstimos Pessoais}: Os clientes podem simular e contratar empréstimos diretamente pelo aplicativo, com taxas personalizadas.
    \item \textbf{NuInvest}: Plataforma de investimentos, que inclui opções de renda fixa e variável, disponível tanto para iniciantes quanto para investidores experientes.
    \item \textbf{Nubank Cripto}: Serviço que permite a compra e venda de criptomoedas diretamente pelo app, com destaque para o crescente portfólio de moedas digitais.
    \item \textbf{NuSeguro}: Inclui seguros de vida e, mais recentemente, seguros para automóveis e residências, com contratação e gestão totalmente digitais.
    \item \textbf{NuPay}: Uma solução de pagamento digital que facilita compras online de forma rápida e segura, sem necessidade de cartão físico.
\end{itemize}

Esses produtos ajudaram o Nubank a se consolidar como uma das maiores fintechs da América Latina, oferecendo serviços inovadores com foco na experiência do usuário \cite{nubank2024}. % Completa e validada
\section{Quais os países nos quais a empresa atua?}

A Microsoft possui operações em mais de 190 países ao redor do mundo, espalhados por todos os continentes:

\begin{itemize}
    \item \textbf{América do Norte:} Canadá; Estados Unidos; México.

    \item \textbf{América Central e Caribe:} Costa Rica; El Salvador; Guatemala; Honduras; Nicarágua; Panamá; Porto Rico; República Dominicana.

    \item \textbf{América do Sul:} Argentina; Bolívia; Brasil; Chile; Colômbia; Equador; Paraguai; Peru; Uruguai; Venezuela.

    \item \textbf{Europa:} Alemanha; Áustria; Bélgica; Bulgária; Chipre; Croácia; Dinamarca; Eslováquia; Eslovênia; Espanha; Estônia; Finlândia; França; Grécia; Holanda; Hungria; Irlanda; Islândia; Itália; Letônia; Lituânia; Luxemburgo; Malta; Noruega; Polônia; Portugal; Reino Unido; República Tcheca; Romênia; Suécia; Suíça.

    \item \textbf{Ásia:} Armênia; Azerbaijão; Bahrain; Bangladesh; Brunei; Camboja; Cazaquistão; China; Coreia do Sul; Emirados Árabes Unidos; Filipinas; Geórgia; Hong Kong SAR; Índia; Indonésia; Iraque; Israel; Japão; Jordânia; Kuwait; Líbano; Malásia; Mongólia; Myanmar; Nepal; Omã; Paquistão; Qatar; Singapura; Sri Lanka; Tailândia; Taiwan; Turquia; Vietnã; Iêmen.

    \item \textbf{África:} África do Sul; Argélia; Egito; Gana; Quênia; Marrocos; Nigéria; Senegal; Tanzânia; Tunísia.

    \item \textbf{Oceania:} Austrália; Nova Zelândia.
\end{itemize}
 % Completa e validada
\section{Quais são os principais concorrentes destas empresas no que se refere ao item 3?}

A Amazon compete em diversos setores com diferentes empresas globais que oferecem produtos e serviços similares. Abaixo estão os principais concorrentes dos produtos e serviços mencionados:

\begin{itemize}
    \item \textbf{Amazon.com (e-commerce)}: A plataforma de comércio eletrônico da Amazon enfrenta concorrência de grandes varejistas online, como o \textbf{Alibaba} (e sua subsidiária AliExpress), o \textbf{eBay} e o \textbf{Walmart}, que têm investido pesado em suas próprias operações de e-commerce \cite{competitors2023ecommerce}.

    \item \textbf{Amazon Prime (streaming e entrega)}: O serviço de assinatura Prime compete diretamente com \textbf{Netflix} e \textbf{Disney+} no setor de streaming de vídeo. No que se refere ao frete rápido e gratuito, o principal concorrente é o serviço \textbf{Walmart+}, que oferece benefícios similares aos membros \cite{streaming2024comparison}.

    \item \textbf{Kindle (leitores de e-books)}: No segmento de leitores de e-books, os principais concorrentes são o \textbf{Kobo} e o \textbf{Nook}, que também oferecem dispositivos dedicados e ecossistemas de leitura digital \cite{kindleCompetitors2024}.

    \item \textbf{Amazon Web Services (AWS) (computação em nuvem)}: No setor de computação em nuvem, o AWS enfrenta forte concorrência da \textbf{Microsoft Azure} e do \textbf{Google Cloud Platform} (GCP), que oferecem serviços similares para empresas que buscam soluções de computação, armazenamento e processamento \cite{awsCompetitors2024}.

    \item \textbf{Alexa (assistente virtual e dispositivos inteligentes)}: No setor de assistentes virtuais, a Alexa compete com o \textbf{Google Assistant}, integrado em dispositivos como o \textbf{Google Nest}, e com o \textbf{Apple Siri}, presente nos dispositivos Apple \cite{alexa2024competitors}.

    \item \textbf{Amazon Fresh e Whole Foods (alimentação)}: A Amazon Fresh e a Whole Foods enfrentam a concorrência de grandes redes de supermercados e entregas de alimentos, como \textbf{Walmart}, \textbf{Kroger} e \textbf{Instacart}, além de serviços emergentes como \textbf{DoorDash} para entrega rápida de mantimentos \cite{groceryCompetitors2024}.

    \item \textbf{Amazon Prime Music (streaming de música)}: No setor de streaming de música, o Amazon Prime Music concorre diretamente com \textbf{Spotify}, \textbf{Apple Music} e \textbf{YouTube Music}, plataformas que possuem uma base de usuários extensa e catálogos globais \cite{musicCompetitors2024}.

    \item \textbf{Amazon Luna (jogos em nuvem)}: No setor de jogos em nuvem, o Amazon Luna compete com o \textbf{Google Stadia} (até sua descontinuação em 2023), \textbf{Microsoft Xbox Cloud Gaming}, e o \textbf{NVIDIA GeForce Now}, que oferecem serviços de jogos acessíveis via streaming \cite{lunaCompetitors2024}.
\end{itemize}

 % Completa e validada
\section{Há proteção intelectual do(s) produto(s) ou processo(s) mais famosos da empresa? Qual o tipo?}

A Petrobras possui um portfólio extenso de patentes, demonstrando seu comprometimento com a inovação tecnológica nos campos de exploração, produção e processamento de petróleo e gás. Dentre as principais patentes registradas estão:

\begin{itemize}
    \item \textbf{Tecnologia de Degradação de PET:} Patente para enzimas capazes de degradar componentes de garrafas PET em menos de um mês, transformando-os em matéria-prima para a indústria petroquímica.
    \item \textbf{Modelagem Digital com Sísmica 4D:} Tecnologia que permite criar modelos tridimensionais de reservatórios de petróleo, adicionando uma dimensão temporal para monitorar mudanças nas reservas.
    \item \textbf{Metodologia para Reduzir Paradas em Unidades de Hidrotratamento:} Inovação tecnológica para aumentar a confiabilidade e eficiência das operações nas refinarias.
\end{itemize}

As patentes protegem as inovações da Petrobras, reforçando sua liderança em tecnologia e sua posição competitiva internacionalmente. Adicionalmente, a Petrobras mantém parcerias com universidades e outras entidades em parcerias estratégicas, visando o desenvolvimento contínuo de novas tecnologias e a expansão de seu portfólio de propriedade intelectual.
 % Completa e validada
\section{Qual o lucro líquido anual mais recente dessa empresa?}

Em 2023, o iFood não apresentou lucro, registrando um \textbf{prejuízo de aproximadamente US\$ 200 milhões} (cerca de R\$ 1 bilhão). Esse resultado se deve aos altos investimentos feitos pela empresa, especialmente em logística, incluindo a expansão de entregas de supermercados e a implementação de "dark stores". Em 2022, a empresa alcançou uma receita de \textbf{R\$ 10 bilhões}, porém o lucro líquido específico desse ano não foi divulgado \cite{tecmundo2023}.

 % Completo e Validado
\section{Esta empresa possui programa(s) de treinamento do seu pessoal? Quais?}

O Nubank oferece diversos programas de treinamento para seus funcionários e estagiários, com foco em desenvolvimento contínuo e capacitação técnica. Um dos principais programas é o \textbf{Programa de Estágio}, que visa estudantes de diversas áreas, especialmente tecnologia, como Engenharia de Software, Análise de Dados e Ciência de Dados. O programa, com duração de 18 meses, segue um modelo de trabalho híbrido e oferece uma trilha de desenvolvimento, incluindo treinamentos, acompanhamento de líderes e feedbacks regulares. Os estagiários têm acesso a benefícios como plano de saúde, suporte psicológico (NuCare) e o programa de aprendizado de idiomas (NuLanguage) \cite{epocanegocios2024, nubankcarreiras2024}.

Além disso, o Nubank lançou recentemente o \textbf{Programa de Formação de Associate Product Managers (APMs)}, que visa capacitar novos profissionais para cargos de gestão de produtos, com foco no desenvolvimento de habilidades funcionais e gerenciais \cite{nubankapm2024}.

 % Completo e Validado
\section{Esta empresa possui programas sociais para as comunidades nas quais atua? Quais?}

Sim, o iFood possui diversos programas sociais que beneficiam as comunidades onde atua:

\begin{itemize}
    \item \textbf{iFood Chega Junto}: Investimento de R\$ 1 milhão em 25 projetos voltados para educação, saúde e segurança de entregadores.
    \item \textbf{iFood Acredita}: Aceleração de empreendedores negros, oferecendo cursos e consultoria.
    \item \textbf{Meu Diploma do Ensino Médio}: Programa que apoia entregadores na conclusão do ensino médio.
    \item \textbf{Potência Tech}: Formação de entregadores em tecnologia \cite{ifoodsocial2023}.
\end{itemize} % Completo e Validado
\section{Esta empresa possui política de preservação do meio ambiente e sustentabilidade? Quais?}

A \textbf{Netflix} adota uma política robusta de preservação ambiental e sustentabilidade como parte de suas práticas de governança ambiental, social e corporativa (ESG). Entre as principais iniciativas, destacam-se:

\begin{itemize}
    \item \textbf{Redução de emissões de carbono}: A Netflix se comprometeu a reduzir suas emissões de carbono em aproximadamente 50\% até 2030. Desde 2022, a empresa compensa as emissões restantes através de investimentos em soluções climáticas naturais verificadas, apoiando as metas globais de emissões líquidas zero \cite{netflix_sustainability}.

    \item \textbf{Sustentabilidade nas produções}: A empresa tem implementado tecnologias sustentáveis em suas produções, como o uso de unidades de energia a hidrogênio e equipamentos elétricos em vez de geradores a diesel. Essas práticas foram testadas em produções como \textit{Bridgerton} e resultaram na redução significativa de emissões de CO2.

    \item \textbf{Investimentos em infraestrutura sustentável}: A Netflix também realiza melhorias em seus estúdios e escritórios, como a instalação de sistemas geotérmicos para aquecimento e resfriamento em seu campus em Albuquerque, com o objetivo de eliminar ou reduzir as emissões das operações prediais.
    
    \item \textbf{Coleção de Histórias Sustentáveis}: A Netflix promove a conscientização ambiental através de seu conteúdo, com uma coleção dedicada a histórias sobre sustentabilidade, que inclui mais de 200 títulos assistidos por milhões de lares ao redor do mundo.

\end{itemize} % Completo e Validado

% ===============================================================
% Conclusão geral ===============================================
% % \begin{titlepage}
\newpage
\setstretch{1.5} % Espaçamento entre linhas, certifique-se de que o pacote setspace está incluído em document.tex

\section{Referências}

\vspace{0.5cm} % Espaço 

\noindent WIKIPEDIA. \textbf{Itautec}. Disponível em: \url{https://pt.wikipedia.org/wiki/Itautec}. Acesso em: 24 ago. 2024.


\noindent TECMUNDO. \textbf{Perdidas no tempo: 10 grandes empresas que estiveram no topo}. Disponível em: \url{https://www.tecmundo.com.br/empresas-e-instituicoes/56240-perdidas-tempo-10-grandes-empresas-estiveram-no-topo.htm}. Acesso em: 24 ago. 2024.


\noindent MORCONE. \textbf{Grandes empresas brasileiras que partiram}. Disponível em: \url{https://morcone.com.br/2024/05/08/grandes-empresas-brasileiras-que-partiram/}. Acesso em: 24 ago. 2024.


\noindent UOL ECONOMIA. \textbf{O que aconteceu com as Lojas Brasileiras, rival das Americanas?}. Disponível em: \url{https://economia.uol.com.br/noticias/redacao/2023/01/29/o-que-aconteceu-com-as-lojas-brasileiras-rival-das-americanas.htm}. Acesso em: 24 ago. 2024.


\noindent UOL ECONOMIA. \textbf{Mappin: do auge à falência}. Disponível em: \url{https://economia.uol.com.br/noticias/redacao/2023/05/30/mappin-do-auge-a-falencia.htm}. Acesso em: 24 ago. 2024.


\noindent ESTADÃO. \textbf{Mappin volta ao jogo do varejo após falência em 1999 e reveses na pandemia, mas marca ainda tem apelo}. Disponível em: \url{https://www.estadao.com.br/economia/negocios/mappin-volta-ao-jogo-do-varejo-apos-falencia-em-1999-e-reves-na-pandemia-mas-marca-ainda-tem-apelo/}. Acesso em: 24 ago. 2024.

% ===============================================================
% Referencias ===================================================
% \begin{titlepage}
\newpage
\setstretch{1.5} % Espaçamento entre linhas, certifique-se de que o pacote setspace está incluído em document.tex

\section{Referências}

\vspace{0.5cm} % Espaço 

\noindent WIKIPEDIA. \textbf{Itautec}. Disponível em: \url{https://pt.wikipedia.org/wiki/Itautec}. Acesso em: 24 ago. 2024.


\noindent TECMUNDO. \textbf{Perdidas no tempo: 10 grandes empresas que estiveram no topo}. Disponível em: \url{https://www.tecmundo.com.br/empresas-e-instituicoes/56240-perdidas-tempo-10-grandes-empresas-estiveram-no-topo.htm}. Acesso em: 24 ago. 2024.


\noindent MORCONE. \textbf{Grandes empresas brasileiras que partiram}. Disponível em: \url{https://morcone.com.br/2024/05/08/grandes-empresas-brasileiras-que-partiram/}. Acesso em: 24 ago. 2024.


\noindent UOL ECONOMIA. \textbf{O que aconteceu com as Lojas Brasileiras, rival das Americanas?}. Disponível em: \url{https://economia.uol.com.br/noticias/redacao/2023/01/29/o-que-aconteceu-com-as-lojas-brasileiras-rival-das-americanas.htm}. Acesso em: 24 ago. 2024.


\noindent UOL ECONOMIA. \textbf{Mappin: do auge à falência}. Disponível em: \url{https://economia.uol.com.br/noticias/redacao/2023/05/30/mappin-do-auge-a-falencia.htm}. Acesso em: 24 ago. 2024.


\noindent ESTADÃO. \textbf{Mappin volta ao jogo do varejo após falência em 1999 e reveses na pandemia, mas marca ainda tem apelo}. Disponível em: \url{https://www.estadao.com.br/economia/negocios/mappin-volta-ao-jogo-do-varejo-apos-falencia-em-1999-e-reves-na-pandemia-mas-marca-ainda-tem-apelo/}. Acesso em: 24 ago. 2024.

% ===============================================================
% Anexo ou Apendices ============================================
% % \begin{titlepage}
\newpage
\setstretch{1.5} % Espaçamento entre linhas, certifique-se de que o pacote setspace está incluído em document.tex

\section{Referências}

\vspace{0.5cm} % Espaço 

\noindent WIKIPEDIA. \textbf{Itautec}. Disponível em: \url{https://pt.wikipedia.org/wiki/Itautec}. Acesso em: 24 ago. 2024.


\noindent TECMUNDO. \textbf{Perdidas no tempo: 10 grandes empresas que estiveram no topo}. Disponível em: \url{https://www.tecmundo.com.br/empresas-e-instituicoes/56240-perdidas-tempo-10-grandes-empresas-estiveram-no-topo.htm}. Acesso em: 24 ago. 2024.


\noindent MORCONE. \textbf{Grandes empresas brasileiras que partiram}. Disponível em: \url{https://morcone.com.br/2024/05/08/grandes-empresas-brasileiras-que-partiram/}. Acesso em: 24 ago. 2024.


\noindent UOL ECONOMIA. \textbf{O que aconteceu com as Lojas Brasileiras, rival das Americanas?}. Disponível em: \url{https://economia.uol.com.br/noticias/redacao/2023/01/29/o-que-aconteceu-com-as-lojas-brasileiras-rival-das-americanas.htm}. Acesso em: 24 ago. 2024.


\noindent UOL ECONOMIA. \textbf{Mappin: do auge à falência}. Disponível em: \url{https://economia.uol.com.br/noticias/redacao/2023/05/30/mappin-do-auge-a-falencia.htm}. Acesso em: 24 ago. 2024.


\noindent ESTADÃO. \textbf{Mappin volta ao jogo do varejo após falência em 1999 e reveses na pandemia, mas marca ainda tem apelo}. Disponível em: \url{https://www.estadao.com.br/economia/negocios/mappin-volta-ao-jogo-do-varejo-apos-falencia-em-1999-e-reves-na-pandemia-mas-marca-ainda-tem-apelo/}. Acesso em: 24 ago. 2024.

% ===============================================================
% End Document ==================================================
\end{document}


\section{Qual é o nome da empresa em estudo? Quando foi fundada? Qual o(s) nome(s) do(s) fundador(es)?}

A Petrobras, oficialmente conhecida como Petróleo Brasileiro S.A., foi fundada em 3 de outubro de 1953, durante o segundo governo de Getúlio Vargas, que é considerado o fundador da empresa. A criação da Petrobras está ligada à Lei nº 2004 de 1953, que estabeleceu o monopólio estatal sobre as atividades de pesquisa, exploração, refino e transporte de petróleo no Brasil. É uma empresa de capital aberto, operando como uma sociedade de economia mista sob controle da União, que detém a maioria do capital votante, garantindo-lhe a gestão e o controle. O governo federal possui uma participação majoritária de aproximadamente 50,26\% das ações ordinárias, enquanto o restante das ações está distribuído entre investidores institucionais, BNDESPar, e o público geral, tanto brasileiros como estrangeiros, através de ações negociadas na B3 e na Bolsa de Nova York.

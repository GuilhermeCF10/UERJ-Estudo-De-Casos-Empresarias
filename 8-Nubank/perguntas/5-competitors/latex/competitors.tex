\section{Quais são os principais concorrentes destas empresas no que se refere ao item 3?}

No Brasil, o Nubank enfrenta competição de vários outros bancos digitais e fintechs, tanto locais quanto globais. Entre seus principais concorrentes estão:

\begin{itemize}
    \item \textbf{C6 Bank}: Um banco digital brasileiro que oferece serviços completos, incluindo contas digitais, cartões de crédito e investimentos, competindo diretamente com o Nubank no Brasil.
    \item \textbf{Banco Inter}: Outro neobanco brasileiro, conhecido por não cobrar taxas de manutenção de contas e oferecer uma plataforma integrada de investimentos.
    \item \textbf{PicPay}: Uma fintech brasileira focada em pagamentos digitais, que também expandiu para oferecer serviços financeiros semelhantes aos do Nubank.
    \item \textbf{Mercado Pago}: Parte do grupo Mercado Livre, o Mercado Pago oferece soluções financeiras e de pagamento digital, sendo um forte competidor para o Nubank no mercado latino-americano.
    \item \textbf{Revolut}: Um banco digital global que começou a atuar no Brasil em 2023, oferecendo serviços como criptomoedas e múltiplas moedas, desafiando o Nubank com sua proposta internacional.
\end{itemize}

Essas fintechs e neobancos estão competindo pelo mesmo público que o Nubank, com foco em serviços financeiros digitais acessíveis e sem tarifas \cite{infomoney2024}.

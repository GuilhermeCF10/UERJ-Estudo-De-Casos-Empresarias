\section{Esta empresa possui programas sociais para as comunidades nas quais atua? Quais?}

Sim, o Nubank possui diversos programas sociais voltados para as comunidades onde atua, sendo o principal deles o \textbf{Instituto Nu}, uma plataforma de inovação social. O Instituto Nu tem como foco principal combater a desigualdade social por meio de três pilares:

\begin{itemize}
    \item \textbf{Educação para empregabilidade}: O Instituto oferece programas de capacitação tecnológica e empregabilidade, como o programa de formação em parceria com a Descomplica, que impacta diretamente jovens em situação de vulnerabilidade social.
    \item \textbf{Empreendedorismo}: O Instituto apoia empreendedores das periferias brasileiras, oferecendo suporte para o desenvolvimento de negócios locais e impulsionando a economia nas favelas.
    \item \textbf{Inovação social}: O Instituto visa democratizar o acesso a recursos filantrópicos e promover iniciativas sociais lideradas por pessoas das favelas, dando visibilidade a esses projetos e ajudando a maximizar seu impacto.
\end{itemize}

Além disso, o Nubank atua com programas de diversidade e inclusão, como o \textbf{NuImpacto}, e oferece suporte financeiro para empreendedores negros por meio do programa \textbf{Semente Preta}. Esses projetos reforçam o compromisso da empresa com o impacto social positivo nas comunidades brasileiras \cite{nubankinstitutonu2024, nubankcarreiras2024, nubankesg2024}.

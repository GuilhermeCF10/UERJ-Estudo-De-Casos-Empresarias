\section{SS, Sociedade Limitada Simples}
A Sociedade Simples é uma forma de associação entre profissionais que exercem predominantemente atividades de natureza intelectual, científica, literária ou artística, com o propósito de oferecer serviços pessoais especializados. Este tipo de sociedade é regido principalmente pelos artigos 982 e 983 do Código Civil Brasileiro de 2002.

\subsection{Características Principais}
\begin{itemize}
    \item \textbf{Natureza não empresarial}: As atividades realizadas têm um caráter mais pessoal, direto e especializado, sem a necessidade de uma estrutura empresarial.
    \item \textbf{Responsabilidade dos sócios}: Dependendo da configuração, os sócios podem ter responsabilidade ilimitada, onde as dívidas sociais podem se estender aos bens pessoais.
    \item \textbf{Administração}: Deve ser realizada por pessoas naturais, com a administração detalhada no contrato social.
    \item \textbf{Registro}: Realizado no Registro Civil de Pessoas Jurídicas, diferentemente das sociedades empresariais que são registradas na Junta Comercial.
\end{itemize}

\subsection{Vantagens e Desvantagens}
\subsubsection{Vantagens}
\begin{itemize}
    \item Flexibilidade para profissionais que desejam colaborar sem complexidades empresariais.
    \item Permite contribuição dos sócios em forma de serviços, além de capital.
\end{itemize}

\subsubsection{Desvantagens}
\begin{itemize}
    \item Maior risco financeiro para os sócios devido à possível responsabilidade ilimitada.
    \item Menos proteção em relação à separação dos bens pessoais dos bens da sociedade.
\end{itemize}

\subsection{Processo de Abertura}
\begin{enumerate}
    \item Elaboração de um contrato social com detalhes claros sobre gestão, responsabilidades e direitos dos sócios.
    \item Registro do contrato no Registro Civil de Pessoas Jurídicas para obtenção do CNPJ.
    \item Obtenção de licenças necessárias para operação conforme as regulamentações locais.
\end{enumerate}

\subsection{Conclusão}
A Sociedade Simples é uma escolha ideal para profissionais que buscam uma cooperação estruturada com flexibilidade e uma abordagem pessoal na prestação de serviços. Embora ofereça vantagens significativas em termos de flexibilidade operacional, é importante considerar as implicações legais e financeiras associadas à sua estrutura de responsabilidade.

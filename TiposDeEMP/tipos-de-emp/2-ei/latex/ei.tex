\section{EI, Empresa Individual}

A Empresa Individual (EI) é uma forma de organização empresarial que permite ao empreendedor atuar como único proprietário, sem a necessidade de formar sociedade. Diferentemente do MEI, a Empresa Individual permite um faturamento maior e não impõe limite de contratação de empregados, mas expõe o empresário a uma maior responsabilidade financeira, uma vez que não há distinção entre o patrimônio pessoal e o empresarial.

\subsection{Características Principais}

As principais características da Empresa Individual incluem:
\begin{itemize}
    \item Não há limite de faturamento anual imposto pela legislação específica do EI.
    \item O empresário responde ilimitadamente pelas dívidas da empresa, ou seja, seu patrimônio pessoal pode ser utilizado para quitações de débitos.
    \item Flexibilidade na gestão e operação do negócio, sem a necessidade de consultas ou aprovações de sócios.
\end{itemize}

\subsection{Benefícios}

Os benefícios de optar por uma Empresa Individual incluem:
\begin{itemize}
    \item Autonomia total na gestão empresarial, com decisões rápidas e sem necessidade de deliberações societárias.
    \item Processo de abertura e encerramento do negócio geralmente mais simples e menos burocrático em comparação com outras formas societárias.
    \item Menos exigências legais e contábeis do que uma sociedade limitada ou uma sociedade anônima.
\end{itemize}

\subsection{Obrigações}

As obrigações de uma Empresa Individual são significativas, especialmente no que tange à responsabilidade por dívidas:
\begin{itemize}
    \item Manutenção de registros financeiros e contábeis atualizados.
    \item Cumprimento de todas as obrigações fiscais e tributárias aplicáveis, dependendo do regime tributário adotado (Simples Nacional, Lucro Presumido ou Lucro Real).
    \item Emissão regular de notas fiscais para todas as transações comerciais.
\end{itemize}

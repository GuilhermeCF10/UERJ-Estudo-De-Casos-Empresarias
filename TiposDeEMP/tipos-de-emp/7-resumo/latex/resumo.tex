\section{Conclusão}

Desse modo, como abordado no Brasil, existem várias formas de estruturação empresarial, cada uma com suas especificidades e vantagens, adequadas para diferentes perfis de empreendedores e negócios, como um resumo podemos deixar que:
\begin{itemize}
    \item \textbf{MEI (Microempreendedor Individual)}:
          \begin{itemize}
              \item Permite a formalização de atividades com faturamento anual de até R\$ 81.000,00.
              \item Oferece benefícios como emissão de notas fiscais e contribuição simplificada ao INSS.
              \item Ideal para pequenos negócios e autônomos que buscam formalização sem grandes custos e burocracia.
          \end{itemize}

    \item \textbf{EI (Empresário Individual)}:
          \begin{itemize}
              \item Não possui limite de faturamento específico.
              \item Oferece maior flexibilidade na contratação de empregados.
              \item Responsabilidade ilimitada do empresário, que responde com seu patrimônio pessoal pelas dívidas da empresa.
          \end{itemize}

    \item \textbf{SLU (Sociedade Limitada Unipessoal)}:
          \begin{itemize}
              \item Oferece segurança da limitação de responsabilidade sem a necessidade de um sócio.
              \item O patrimônio pessoal é protegido.
              \item A empresa pode operar com a flexibilidade de uma sociedade limitada, mas com um único sócio.
          \end{itemize}

    \item \textbf{LTDA (Sociedade Limitada)}:
          \begin{itemize}
              \item Ideal para empresas com mais de um sócio.
              \item O capital social é dividido em cotas, e a responsabilidade dos sócios é limitada ao valor dessas cotas.
              \item Oferece proteção patrimonial e uma gestão coletiva com segurança jurídica.
          \end{itemize}

    \item \textbf{SS (Sociedade Simples)}:
          \begin{itemize}
              \item Voltada para profissionais de atividades intelectuais, científicas, literárias ou artísticas.
              \item Permite a união de esforços sem o rigor empresarial.
              \item A responsabilidade dos sócios pode ser ilimitada, dependendo da configuração.
              \item Registro realizado no Registro Civil de Pessoas Jurídicas, não na Junta Comercial.
          \end{itemize}

    \item \textbf{SA (Sociedade Anônima)}:
          \begin{itemize}
              \item Ideal para grandes negócios que buscam captar recursos por meio de ações.
              \item O capital é dividido em ações, e a responsabilidade dos acionistas é limitada ao valor das ações subscritas.
              \item Pode ser de capital aberto, com ações negociadas em bolsa, ou de capital fechado, com ações restritas a um grupo seleto de investidores.
          \end{itemize}

\end{itemize}
\vspace{0.4cm}
Este progresso desde o MEI até a Sociedade Anônima mostra a diversidade de opções disponíveis para empreendedores no Brasil, permitindo que cada negócio escolha a estrutura que melhor se adapta às suas necessidades e objetivos de crescimento.

\vspace{0.4cm}
Além desses, também existe o \textbf{Inova Simples (IS)}, um regime especial criado pela Lei Complementar nº 167/2019, destinado a startups e empresas de inovação. Este regime facilita a criação, formalização e desenvolvimento de negócios inovadores, permitindo que projetos em fase de testes possam ser formalizados de maneira simplificada, com processos mais ágeis de registro e tributação. 

Um ponto importante é que o Inova Simples permite a captação de recursos ou realização de comercializações experimentais de até R\$ 81.000,00, proporcionando uma estrutura simplificada para o desenvolvimento inicial de startups. No entanto, a empresa deve se adequar a um regime tributário caso suas atividades ultrapassem esse limite, ou quando estiver pronta para sair da fase experimental. O Inova Simples é uma alternativa ideal para empreendedores que desejam explorar ideias inovadoras com um caráter disruptivo ou incremental, contribuindo para o fomento da inovação no Brasil.

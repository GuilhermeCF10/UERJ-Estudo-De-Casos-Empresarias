\section{MEI, Microempreendedor Individual}

O Microempreendedor Individual (MEI) constitui uma categoria empresarial no Brasil projetada para regularizar a atuação de pequenos empreendedores que, frequentemente, operam de maneira autônoma e informal. Essa categoria facilita a formalização desses empreendedores como empresas através de um processo simplificado e econômico. O objetivo principal da implementação do MEI é promover a inclusão desses trabalhadores autônomos no sistema econômico formal, proporcionando uma estrutura que permite a transição do informal para o formal com mínimas barreiras burocráticas e financeiras.

\subsection{Características Principais}

Algumas das principais características do MEI incluem:
\begin{itemize}
    \item Faturamento máximo permitido de até R\$ 81.000,00 por ano.
    \item Não é permitida a participação em outra empresa como sócio ou titular.
    \item O MEI pode ter no máximo um empregado contratado que receba o salário mínimo ou o piso da categoria.
    \item É obrigatório contribuir mensalmente para o INSS, garantindo assim benefícios como aposentadoria e auxílio-doença.
\end{itemize}

\subsection{Benefícios}

Entre os benefícios de ser um MEI, estão:
\begin{itemize}
    \item Emissão de nota fiscal eletrônica.
    \item Acesso a serviços bancários, incluindo crédito.
    \item Baixo custo de tributação, com valores fixos mensais que cobrem tributos como INSS e ISS ou ICMS, dependendo do tipo de serviço ou produto fornecido.
    \item Possibilidade de participar de licitações públicas.
    \item O pagamento do imposto mensal, que em 2024 está em cerca de R\$ 71,60, é considerado baixo em comparação com outros regimes tributários e já conta como tempo de contribuição para a aposentadoria.
\end{itemize}

\subsection{Obrigações}

As obrigações do MEI incluem:
\begin{itemize}
    \item Realização de uma declaração anual simplificada.
    \item Manutenção de um registro das vendas e compras realizadas.
    \item Emissão de nota fiscal quando o cliente for outra empresa.
\end{itemize}